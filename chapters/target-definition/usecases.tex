
\subsection{Nutzungsszenarien}

Die folgenden Nutzungsszenarien sollen zeigen wie mit dieser Anwendung umgegangen werden können soll.
Sie Beschreiben exemplarische Verwendung, bei der ein konkreter Vorgang zu neuen Erkenntnissen bzw. Resultaten führen soll.

Beide Szenarien sind iterativ zu verstehen, der Nutzer soll diese Aktion wiederholen um sich sukzessiv dem gewünschten Ziel bzw. Verhalten zu nähern.

\textit{USE - use case - Nutzungsszenario}


%\subsubsection{USE\subsectionid{} - Abschätzen von Verkehrssystemänderungen}
%
%Ein Nutzer möchte durch das System in der Lage sein, die Auswirkungen einer Veränderung an einem Straßennetz abzuschätzen.
%
%Dafür definiert er zuerst ein neues Straßenlayout, durch welches er den für ihn interessanten Sachverhalt modelliert.
%Im Anschluss daran wählt er einen Programm aus, welches in dieser Welt Fahrzeuge verwalten soll.
%
%Der Nutzer verwendet dann die Anwendung um den aktuellen Simulationszustand einzusehen.
%Diese Darstellung verfolgt er in Echtzeit um die Auswirkung seiner Veränderungen an dem Straßenlayout abschätzen zu können.
%
%Im Anschluss plant der Nutzer eine Veränderung am Verkehrssystem um dem gewünschten Resultat näher zu kommen oder um die Stärke der Auswirkung besser Einschätzen zu können.

\subsubsection{USE\#01 - Erprobung eines Agenten}

Bei der Erprobung des Agenten beginnt der Nutzer mit der Erstellung eines solchen.
Dafür zieht er die Schnittstellenbeschreibung der Plattform heran und plant die Eigenschaften seines Agenten.

Nach einer ersten Implementation übergibt der Nutzer den Quellcode dem System.
Im Anschluss erstellt er eine neue Simulation mit diesem Agenten und einem Verkehrslayout.

Diese Simulation sieht er dann in Echtzeit ein, um nachvollziehen zu können welches Verhalten dieser Agent erzeugt.
Der Nutzer nutzt dafür eine plastische Darstellung des Simulationszustandes mit welcher er interagieren kann, indem er die die Kamera bewegt.

Zusätzlich kann der Nutzer auf Fahrzeuge oder andere Element klicken um über diese Informationen zu ermitteln welche der Agent bereitstellt.
Dies erlaubt ihm einzuschätzen warum der Agent dieses Verhalten zeigt und erlaubt es ihm somit Veränderungen zu planen und anzuwenden.
