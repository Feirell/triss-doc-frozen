\newpage

\section{Fazit}

Insgesamt ist dieses Projekt ein Erfolg.
Es ist gelungen, eine Test-Implementation einer Simulations-Plattform für die Echtzeitdarstellung von Simulationszuständen eines Agenten im Kontext der Verkehrsdomäne zu erschaffen.

Abschließend lässt sich sagen, dass die initiale Fragestellung, ob sich solch eine Agenten-Simulationsplattform mit dem JavaScript Ökosystem realisieren lässt, mit \enquote{Ja} beantwortet werden kann.

Die angestrebte Plattform ließ sich nicht nur im Hinblick auf die nicht funktionalen Anforderungen mit diesen Technologien realisieren, sondern konnte darüber hinaus auch die angestrebte hohe Performance erreichen, die selbst die Implementation und Auswertung von mittelgroßen Agenten mithilfe einer Echtzeit-3D-Darstellung erlaubt.

Es muss allerdings auch festgehalten werden, dass es noch viele Wege gäbe, die Anwendung zu erweitern und zu verbessern.
Vor allem durch den Punkt \refsec{sec:fnc-impr-local-computation} ließe sich die Verwendung bedeutend weiter vereinfachen, sodass noch mehr Menschen von ihr profitieren könnten.

Insgesamt ließ sich jedoch eine sehr solide Plattform erschaffen, die bereits jetzt einen relevanten Mehrwert bietet und auch ohne weitere Anpassungen für unterschiedliche Anwender und andere interessierte Entwickler als Beispiel-Implementation nützlich sein kann.
