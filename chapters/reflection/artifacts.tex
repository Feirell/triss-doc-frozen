\subsection{Quellcode und Artefakte} \label{sec:artefacts}

Die durch diese Entwicklung entstandenen Quelldateien, die Quelldateien für dieses Dokument, die Scripte zur Erschaffung der Artefakte sowie die vorgebauten Artefakte sind alle online frei zugänglich verfügbar.

Der gesamte Quellcode ist unter der GPL 3.0 Lizenz als Git-Repository auf GitHub verfügbar.
Dabei findet man in dem TRISS-Application Repository\footnote{\url{https://github.com/Feirell/triss-app-frozen}} die Quelldateien für alle Programm-Artefakte, unter dem TRISS-Documententation Repository\footnote{\url{https://github.com/Feirell/triss-doc-frozen}} sind die Quelldateien für dieses Dokument enthalten und im Dockerhub als \enquote{trissapp/triss-server}\footnote{\url{https://hub.docker.com/r/trissapp/triss-server}} das Backend bzw. als \enquote{trissapp/triss-client}\footnote{\url{https://hub.docker.com/r/trissapp/triss-client}} das Frontend.

Im TRISS-Application Repository\footnote{\url{https://github.com/Feirell/triss-app-frozen}} findet man neben dem Quellcode auch eine technische Einführung in das Projekt sowie eine kurze Anleitung für die Entwicklung eines Agenten.

In dieser technischen Einleitung findet man auch eine generelle Einführung in die Quellcode-Struktur und wie diese angepasst werden kann.
Zusätzlich ist dort auch beschrieben, wie sich die Artefakte in Betrieb nehmen lassen.

Darüber hinaus wurden im Kontext der Arbeit einige Artefakte erstellt, die einen generellen unabhängigen Mehrwert hatten und deshalb in das öffentliche npm-Repository ausgelagert worden sind.

Dazu zählen primär die folgenden drei Pakete.

Der \textit{performance-test-runner}\footnote{\url{https://npmjs.org/package/performance-test-runner}} ist ein Paket, das verwendet worden ist, um unterschiedliche Implementationen gegeneinander abzuwägen und somit stringent auf Verbesserungen hinzuarbeiten.

Bei dem Paket \textit{event-emitter-typesafe}\footnote{\url{https://npmjs.org/package/event-emitter-typesafe}} handelt es sich um eine Teil-Implementation des DOM-Standards für EventEmitter\autocite{domSpecEvent2022}, welche hier allerdings Typsicher implementiert worden ist und zusätzlich nicht die Runtime-Implementation benötigt und somit auch für Node.js Anwendung verwendet werden konnte.

Ebenso wurde das wichtige \textit{serialization-generator}\footnote{\url{https://npmjs.org/package/serialization-generator}} Paket ausgelagert.
Dieses ist, wie bereits zu erkennen war, von zentraler Bedeutung, da es die performante Übertragung der Simulationszustände zwischen den Realms erlaubt.
Hinzu kommt, dass darauf Wert gelegt worden ist, die Spezifikation der Datenstrukturen bzw. der (De-)Serialisierer sehr leserlich und intuitiv zu halten.
