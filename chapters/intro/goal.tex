\subsection{Ziel}

Das Ziel dieser Arbeit soll es sein, zu ermitteln, ob mithilfe des JavaScript Ökosystems eine Anwendung in einer Client Server Architektur erstellt werden kann, die performant mehrere tausend Fahrzeuge, die durch eine agentenbasierte Modellierung kontrolliert werden, vom Server zum Client übertragen und dreidimensional darstellen kann.

Dafür soll eine solche Test-Anwendung entwickelt und im Anschluss ihre Performanz eingeschätzt werden.
Diese Anwendung soll dafür dem Nutzer die Möglichkeit bieten, mithilfe einer einfachen Schnittstelle einen Agenten zu spezifizieren und innerhalb der Plattform betreiben zu lassen.
Zusätzlich soll der Nutzer in der Lage sein, über eine einfache grafische Schnittstelle ein Straßennetz zu definieren, das er mit dem Agenten zusammen zu einer laufenden Instanz kombinieren kann.
Diese soll dann in Echtzeit anschaulich und plastisch dargestellt werden können.
