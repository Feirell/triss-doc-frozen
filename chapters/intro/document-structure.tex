\subsection{Struktur der Arbeit}

Die hier folgende Arbeit soll die aufgebrachte Zielsetzung und deren Umsetzung beschrieben.
Dafür ist sie in drei Arbeitsschritte bzw. deren Kapitel geteilt.

Begonnen wird bei der Definition der Anforderungen.
Dabei wird zwischen den Funktionalen- und nicht Funktionalen-Anforderungen unterschieden.

Im Anschluss folgt die Beschreibung der Umsetzung in Form des angestrebten Konzeptes.
Dieses untergliedert sich zuerst in die Lösungsstrategie, die grundsätzlich beschreiben soll, welche Struktur die übergeordnete Lösung haben soll.

Darauf folgen die querschnittlichen Konzepte, welche sich vor allem auf die eingesetzten Technologien, ihre Auswahlkriterien so wie deren Eigenschaften beziehen.
Die sich daran anschließende Machbarkeitsvalidierung half dabei einschätzen zu können, ob die Plattform überhaupt machbar war.

Die sich dem anschließenden drei Kapitel gehen dann ins Detail und stellen erst aus Sicht der Bausteine, dann aus dem Blick der Laufzeit und schlussendlich im Sinne der Verteilung die Komponenten und deren Interaktionen dar.
Sie sollen durch sukzessive präzisierung dabei helfen nachvollziehen zu können wie die Anwendung strukturiert ist, wie sie intern funktioniert und welche Überlegungen dabei angestellt worden sind.

Schlussendlich soll innerhalb der Konzipierung noch der Beispielagent und sein Verhalten dargestellt werden.
Als Beispiel wird er hier aufgeführt damit durch ihn erschlossen werden kann, wie strukturell auch dieser Bestandteil funktionieren kann, welche Aspekte dabei zu beachten sind aber auch wie einfach dessen Implementation sein kann.

Die Resultatsbetrachtung stellt dann den Abschluss dar und erlaubt eine gute Übersicht über die Anwendung.
In diesem Kapitel werden dann die Anforderungen evaluiert und eingeschätzt welche Leistungsmerkmale die Plattform nun tatsächlich aufweist um damit auch die Beantwortung der initialen Fragestellung bzw. die Zielerfüllung abschätzen zu können.

Der Ausblick soll danach aber auch aufzeigen, welche Aspekte diese Plattform noch nicht bereitstellt und durch welche Erweiterungen sie noch an Funktionalität bzw. Leistung gewinnen könnte.
Das Fazit fasst dann das Arbeitsergebnis noch einmal zusammen und beantwortet die eingehende Frage abschließend.
